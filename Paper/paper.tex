\documentclass[conference]{IEEEtran}
\IEEEoverridecommandlockouts
% The preceding line is only needed to identify funding in the first footnote. If that is unneeded, please comment it out.
\usepackage{cite}
\usepackage{amsmath,amssymb,amsfonts}
\usepackage{algorithmic}
\usepackage{graphicx}
\usepackage{textcomp}
\usepackage{xcolor}
\def\BibTeX{{\rm B\kern-.05em{\sc i\kern-.025em b}\kern-.08em
    T\kern-.1667em\lower.7ex\hbox{E}\kern-.125emX}}
\begin{document}

\title{Caching in Named Data Networking for the Wireless Internet of Things}

\author{\IEEEauthorblockN{Merlin Steuer}
\IEEEauthorblockA{Universität zu Lübeck}\\
Lübeck, Germany \\
merlin.steuer@student.uni-luebeck.de
}

\maketitle

\begin{abstract}
\end{abstract}

\begin{IEEEkeywords}
    Named Data Networking, Information-Centric
    Networking, Internet of Things, In-network caching, Freshness
\end{IEEEkeywords}

\section{Introduction}

The technologies the internet currently relies on were built for requirements dating a long while back. Back then, relatively few computers needed to be connected to exchange relatively large, static documents or files. Nowadays, however, the structure and thus the requirements for the interconnection of devices has changed. Within the last few decades, the Internet of Things (IoT) has brought up many new types of interconnected devices and new requirements for technologies. One important thing that has changed is the type of data that is produced and consumed in networks. In IoT applications, data packets are usually small and transient, for example, measurements produced by a wireless sensor.

In the paper the researchers focussed on wireless devices, which bring a multitude of special requirements and limitations with them. On one hand, wireless devices have very limited energy resources, making efficient transporting of data an important factor. On the other hand, these devices usually have very small memory sizes and are thus not capable of storing large amounts of data.

The researchers propose a new caching strategy for Named Data Networks, {pCASTING}, which is especially suited for applications in wireless IoT networks. It could be shown that the strategy is beneficial for both energy consumption and memory usage while also increasing data availability and decreasing response times to data requests.

The content of this short paper is as follows. In section~\ref{sec:background} a short introduction to Named Data Networking and caching strategies is given. In Section~\ref{sec:pcasting} the proposed caching strategy is discussed in detail. In Section~\ref{sec:eval} the strategy is evaluated and the paper is concluded in Section~\ref{sec:conclusion}.

\section{Background}
\label{sec:background}

\subsection{Named Data Networking}

\subsection{Caching in Named Data Networking}

\section{The pCASTING Caching Strategy}
\label{sec:pcasting}

\section{Evaluation}
\label{sec:eval}

\section{Conclusion}
\label{sec:conclusion}

\end{document}
